%%%%%%%%%%%%%%%%%%%%%%preface.tex%%%%%%%%%%%%%%%%%%%%%%%%%%%%%%%%%%%%%%%%%
% sample preface
%
% Use this file as a template for your own input.
%
%%%%%%%%%%%%%%%%%%%%%%%% Springer %%%%%%%%%%%%%%%%%%%%%%%%%%
% \Extrachap{Preface}
\preface

%% Please write your preface here
\indent This book consists of two separate, but closely related, parts. The first part (Chapters $1-10$) is subtitled 
\textit{The Elements of Integration;} the second part (Chapters $11-17$) is subtitled \textit{The Elements of Lebesgue measure.} 
It is possible to read these two parts in either order, with only a bit of repetition.\\
\indent The Elements of Integration is essentially a corrected reprint of a book with that title, originally published in 1966, 
designed to present the chief results of the Lebesgue theory of integration to a reader having only a modest mathematical background. 
This book developed from my lectures at the University of Illinois, Urbana-Champaign, and it was subsequently used there and elsewhere 
with considerable success. Its only prerequisites are a understanding of elementary real analysis and the ability to comprehend ``$\varepsilon - \delta$ 
arguments''. We suppose that the reader has some familiarity with the Riemann integral so that it is not necessary to provide 
motivation and detailed discussion, but we do not assume that the reader has a mastery of the subtleties of that theory. A solid 
course in ``advanced calculus'', an understanding of the first third of my book \textit{The Elements of Real Analysis,} or of most 
of my book \textit{Introduction to Real Analysis} with D. R. Sherbert provides an adequate background. In preparing this new edition, 
I have seized the opportunity to correct certain errors, but I have resisted the temptation to insert additional material, since I 
believe that one of the features of this book that is most appreciated is its brevity.\\
\noindent \textit{The Elements of Lebesgue Measure} is descended from class notes written to acquaint the reader with the theory 
of Lebesgue measure in the space $\textbf{R}^p$. While it is easy to find good treatments of the case $p=1$, the case $p>1$ is not 
quite as simple and is much less frequently discussed. The main ideas of Lebesgue measure are presented in detail in Chapters $10-15$, 
although some relatively easy remarks are left to the reader as exercises. The final two chapters venture into the topic of 
nonmeasurable sets and round out the subject.\\
\indent There are many expositions of the Lebesgue integral from various points of view, but I believe that the abstract measure 
space approach used here strikes directly towards the most important results: the convergence theorems. Further, this approach is 
particularly well-suited for students of probability and statistics, as well as students of analysis. Since the book is intended as 
an introduction, I do not follow all of the avenues that are encountered. However, I take pains not to attain brevity by leaving out 
important details, or assigning them to the reader.\\
\indent Readers who complete this book are certainly not through, but if this book helps to speed them on their way, it has accomplished 
its purpose. In the References, I give some books that I believe readers can profitably explore, as well as works cited in the body of the text.\\
\indent I am indebted to a number of colleagues, past and present, for their comments and suggestions; 
I particularly wish to mention N. T. Hamilton, G. H. Orland, C. W. Mullins, A. L. Peressini, and J. J. Uhl, Jr. I also wish to thank 
Professor Roy O. Davies of Leicester University for pointing out a number of errors and possible improvements.\\


\begin{flushright}\noindent

  \hfill {\bf ROBERT G. BARTLE}\\
\end{flushright}
\textit{Ypsilanti and Urbana} \\ \textit{Novenber 20, 1994}\\
